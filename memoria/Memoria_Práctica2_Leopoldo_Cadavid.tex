\documentclass[12pt, a4paper]{article}

\usepackage{amsmath}
\usepackage[utf8]{inputenc}
\usepackage[spanish]{babel} %Paquete de idioma
\usepackage[hidelinks]{hyperref}
\usepackage{graphicx}
\usepackage{float}
\usepackage{eso-pic}
\usepackage{lipsum}
\usepackage{transparent}
\usepackage{parskip}
%\usepackage[backend=biber, style=apa]{biblatex}

\graphicspath{{imagenes/}}

%%%%%%%%% ESTO PARA LA MARCA DE AGUA %%%%%%%%%%%%%%%%%%%%%%%%%%%%%%%


\AddToShipoutPicture{ 
    \put(410,380){
        \parbox[b][\paperheight]{\paperwidth}{%
            \vfill
            \
            {
            \transparent{1}
            \includegraphics[scale=0.5]{logo-ua.png}
            \vfill
            }
        }
    }
}
%%%%%%%%%%%%%%%%%%%%%%%%%%%%%%%%%%%%%%%%%%%%%%%%%%%%%%%%%%

\title{Memoria Práctica 2 \\


\large Comunicaciones
}
\author{
Leopoldo Cadavid Piñero
}
\date{Febrero 2022}







\begin{document}

\maketitle
\newpage
\tableofcontents
\newpage
\section{Introducción}
      

\section{Sesión 1}

\section{Sesión 2}
 En esta sesión se plantean cómo objetivos:
 \begin{itemize}
     \item Terminar con las pruebas del programa de la sesión 1, Ubidots
     \item Iniciar el porceso de automatización de una sencilla instalación, usando protocolos de 
        comunicación vistos en la teoría.
    
    
    \end{itemize}

    \subsection{Continuación del uso de Ubidots}

\subsection{Iniciación de control automatico}

Pasos:
\begin{itemize}
    \item Conectarse a la red del routeer de la instalación que es la subred de control
    \item meterse en la  interfaz de shelly
    \item Desde aquí podremos conectarnos a los dispositivos del sistema y probar como podemos dar instrucciones y recibir info
    \item \textbf{Aconsejable:} decirle al shelly a que red se tiene que conectar. Decimos a shelly que se conecte a la red que nosotros queremos
    (esto se graba en su memoria).
    \item Para nuestro caso: \textbf{Nombre de la red:} Cudy-081c, \textbf{IP:} 192.168.10.100 (rango x.x.10.(0/255)), 
    \textbf{Gateway (opcional):}. 
    \item Si hemos introducido todo bien, nuestro shell estará conectado en la red del router.
    \item Necesitamos saber las peticiones get para acceder a los compornentes.
    \item En clase el compañero se ha conectado a la ip comentada y aha encendido y apagado el relé de la instalación.
    

\end{itemize}

\section{Sessión 3}



El profesor nos da especificaciones: 
\begin{itemize}
    \item \textbf{Red:} Cudy-081c ,  \textbf{Clave:} Comunica2022
    \item Nos conectamos a la IP: 192.168.10.100
    \item \textbf{Peticiones:} nos metemos en Nodered y con el 
    nodo \textit{http request} con el método \textbf{get y url 192.168.10.110/relay/0?turn=off}. Con esto, nos conectaremos al relé de 
    la estación de clase y podemos encender o apagar el relé según consideremos. 
 
\end{itemize}



\section{Sesión 4}

La idea de esta sesión es cambiar el uso de http requests, por 
el uso de nodos mqtt, para que la carga de las conexiones recaiga sobre
el broker, en este caso la raspberry, en vez de en el shelly.

\begin{itemize}
    \item creamos nodo mqtt
    \item Debemos configurar info del broker: Clave: comunicacionespass;
    IP: 192.16810.144
    \item configuramos el topic según los intereses 
\end{itemize}

El siguiente paso de la sesión será ver como podemos ver el topic del
shelly donde se estan publicando los datos de la potencia.
\end{document}

