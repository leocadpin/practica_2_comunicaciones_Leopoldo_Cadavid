\documentclass[12pt, a4paper]{article}

\usepackage{amsmath}
\usepackage[utf8]{inputenc}
\usepackage[spanish]{babel} %Paquete de idioma
\usepackage[hidelinks]{hyperref}
\usepackage{graphicx}
\usepackage{float}
\usepackage{eso-pic}
\usepackage{lipsum}
\usepackage{transparent}
\usepackage{parskip}
%\usepackage[backend=biber, style=apa]{biblatex}

\graphicspath{{imagenes/}}

%%%%%%%%% ESTO PARA LA MARCA DE AGUA %%%%%%%%%%%%%%%%%%%%%%%%%%%%%%%


\AddToShipoutPicture{ 
    \put(410,380){
        \parbox[b][\paperheight]{\paperwidth}{%
            \vfill
            \
            {
            \transparent{1}
            \includegraphics[scale=0.5]{logo-ua.png}
            \vfill
            }
        }
    }
}
%%%%%%%%%%%%%%%%%%%%%%%%%%%%%%%%%%%%%%%%%%%%%%%%%%%%%%%%%%

\title{Memoria Práctica 2 \\


\large Comunicaciones
}
\author{
Leopoldo Cadavid Piñero
}
\date{Febrero 2022}







\begin{document}

\maketitle
\newpage
\tableofcontents
\newpage
\section{Introducción}
      

\section{Sesión 1}

\section{Sesión 2}
 En esta sesión se plantean cómo objetivos:
 \begin{itemize}
     \item Terminar con las pruebas del programa de la sesión 1, Ubidots
     \item Iniciar el porceso de automatización de una sencilla instalación, usando protocolos de 
    comunicación vistos en la teoría.
    \end{itemize}













\end{document}